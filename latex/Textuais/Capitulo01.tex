\section*{Uma Nova Abordagem para o Desenvolvimento Econômico}

As instituições desempenham um papel crucial no desenvolvimento econômico de longo prazo, funcionando como o arcabouço que regula o comportamento das empresas, dos indivíduos e do governo \cite{}. Sob a perspectiva Schumpeteriana, a inovação tecnológica e o dinamismo empresarial são motores centrais do crescimento econômico. No entanto, as interações entre governança e instituições, especialmente no contexto de economias em desenvolvimento, podem tanto fomentar quanto restringir esse processo. Esta pesquisa busca examinar como a qualidade das instituições políticas e econômicas afeta o desempenho econômico, com especial atenção para a relação entre conexões políticas, inovação e destruição criativa \cite{artigo1-SchumpeterianGrowthModel}.

Inspirado pelas ideias de Schumpeter, que enxergava o capitalismo como um processo contínuo de destruição criativa, onde novas tecnologias substituem as antigas \cite{Freitas2007UmaDD}, este estudo analisa como as dinâmicas econômicas são impactadas pela eficiência institucional. Vamos explorar dados empíricos relacionados ao mercado de trabalho, às finanças empresariais e à inovação tecnológica, com foco em como essas variáveis se relacionam com a governança e as instituições políticas \cite{Felipe2008InstituiesEM}.

A pesquisa também aborda o papel das conexões políticas nas empresas e suas implicações para a produtividade e inovação. Fatores como a facilidade em superar barreiras burocráticas e regulatórias, o acesso a recursos e a posição política de líderes empresariais são analisados a partir de um vasto conjunto de dados provenientes de diversas fontes institucionais \cite{Filho_Filho_2022}.

\section*{O Caminho da Análise Empírica}

\subsection*{Fontes de Dados}

Para compreender a interação entre instituições, governança e desenvolvimento econômico, utilizaremos um conjunto robusto de dados de diferentes origens. As fontes de dados incluem informações do sistema de seguridade social, dados financeiros de empresas, dados sobre patentes e registros de políticos locais, além de resultados eleitorais \cite{Pond2005InstituiesEM}. Esses dados abrangem um período de aproximadamente três décadas, proporcionando uma visão detalhada das dinâmicas econômicas e políticas em uma economia em desenvolvimento.

\begin{itemize}
    \item \textbf{Dados da Seguridade Social (INPS)}: O acesso a dados de seguridade social italianos permite a análise de um universo de trabalhadores do setor privado, excluindo contratados e trabalhadores agrícolas. Esses dados fornecem informações completas sobre o histórico de emprego, renda laboral e características demográficas dos trabalhadores. A partir dessas informações, agregamos os dados para criar variáveis sobre o tamanho das empresas, salários médios e composição da força de trabalho.
    
    \item \textbf{Dados Financeiros das Empresas (Cerved)}: Dados financeiros de empresas incorporadas na Itália fornecem acesso a balanços e demonstrações de resultados, permitindo calcular a produtividade do trabalho e a produtividade total dos fatores. Isso possibilita a comparação entre empresas com e sem conexões políticas, e como tais conexões influenciam a dinâmica financeira e a performance econômica.
    
    \item \textbf{Dados de Patentes (PATSTAT)}: Dados do Escritório Europeu de Patentes oferecem informações detalhadas sobre todas as patentes registradas por empresas italianas. Essas informações nos permitem medir a inovação em nível de firma, considerando tanto a quantidade quanto a qualidade das patentes, usando indicadores como o número de pedidos de patentes e as citações recebidas.
    
    \item \textbf{Registros de Políticos Locais e Dados Eleitorais (Ministério do Interior)}: Informações sobre políticos locais e resultados eleitorais complementam a análise, permitindo identificar empresas conectadas a políticos e como essas conexões influenciam o desempenho dessas empresas. O registro inclui detalhes sobre a afiliação partidária, cargos ocupados e a posição hierárquica dos políticos \cite{martes_2010}.
\end{itemize}

\subsection*{Metodologia}

A metodologia proposta baseia-se em uma combinação de técnicas econométricas para analisar a influência das variáveis institucionais sobre o desempenho econômico das empresas \cite{artigo1-SchumpeterianGrowthModel}. Utilizaremos um modelo de regressão para estimar o impacto das conexões políticas e da qualidade das instituições na produtividade, inovação e sobrevivência das empresas ao longo do tempo. A produtividade do trabalho e a inovação serão analisadas como variáveis dependentes, enquanto as variáveis explicativas incluirão características das firmas, dados demográficos e a presença de conexões políticas.

Além disso, será empregada uma análise de descontinuidade em regressão (RDD) para explorar eleições marginais, identificando os efeitos causais das conexões políticas sobre o desempenho das firmas. A utilização de dados eleitorais detalhados, combinados com informações sobre empresas e patentes, permitirá uma abordagem robusta e abrangente da questão \cite{Freitas2007UmaDD}.

A análise será complementada com o uso do \emph{Replication Package} do artigo-base, o que nos permitirá realizar uma replicação e expansão dos resultados, ajustando o modelo para incorporar o contexto institucional analisado nesta pesquisa.
