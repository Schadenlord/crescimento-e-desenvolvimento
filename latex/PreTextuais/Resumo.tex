% ---
% RESUMOS
% ---

% resumo em português
\setlength{\absparsep}{18pt} % ajusta o espaçamento dos parágrafos do resumo
\begin{resumo}
    O artigo "Sistemas políticos, memória de regime e liberdade econômica" de Peter Calcagno, Beatriz Maldonado, Todd Nesbit e Mary Frances Zeager investiga a influência da memória de regimes passados na liberdade econômica de um país. Os autores desenvolvem uma medida inovadora de memória de regime e analisam o efeito geracional de regimes anteriores na liberdade econômica. Por meio de um estudo com 144 países entre 1970 e 2015, o artigo destaca que a memória de regime pode promover melhorias na liberdade econômica em nações historicamente democráticas, enquanto desencoraja em países com histórico autocrático. Esses resultados contribuem para a compreensão de como a cultura, as tradições e as instituições passadas impactam as políticas econômicas atuais.

 \textbf{Palavras-chave}: Sistemas políticos. Memória de regime. Liberdade Econômica. Cultura. Instituições.
\end{resumo}
